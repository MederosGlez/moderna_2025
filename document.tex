\documentclass{article}
\usepackage{amsmath, amssymb, booktabs, array}
\usepackage[spanish]{babel}
\usepackage[utf8]{inputenc}
\usepackage{enumitem}
\usepackage{float}

\title{Resumen Física Moderna}
\author{}
\date{}

\begin{document}
	\maketitle
	
	\section{Radiación Térmica y Postulado de Planck}
	\subsection{Deducción de la Fórmula de Rayleigh-Jeans (enfoque clásico)}
	
	\textbf{Objetivo}: Relacionar la geometría de la cavidad con las frecuencias permitidas. Para una cavidad cúbica de lado \(a\), las ondas estacionarias deben tener nodos (amplitud cero) en las paredes \(x=0\), \(x=a\), \(y=0\), \(y=a\), \(z=0\), \(z=a\).

	Para una onda propagándose en una dirección arbitraria, se tienen los cosenos directores \(\cos\alpha\), \(\cos\beta\), \(\cos\gamma\):
	
	\begin{equation}
		\lambda_x = \frac{\lambda}{\cos\alpha},\quad \lambda_y = \frac{\lambda}{\cos\beta},\quad \lambda_z = \frac{\lambda}{\cos\gamma}
	\end{equation}
	
	La proyección de la longitud de onda total \(\lambda\) en cada eje define la distancia entre nodos en esa dirección.
	
	Como en \(x = a\) debe haber un nodo:
	\begin{equation}
		\frac{a}{\lambda_x/2} = n_x \Rightarrow \frac{2a}{\lambda_x} = n_x \quad (n_x \in \mathbb{N}^+)
	\end{equation}
	
	Sustituyendo \(\lambda_x\):
	\begin{equation}
		\frac{2a}{c}\nu\cos\alpha = n_x
	\end{equation}
	
	Similar para \(y\) y \(z\). Los enteros \(n_x, n_y, n_z\) determinan el número de nodos en cada dirección.
	
	Elevando al cuadrado y sumando:
	\begin{equation}
		\left(\frac{2a}{c}\nu\right)^2(\cos^2\alpha + \cos^2\beta + \cos^2\gamma) = n_x^2 + n_y^2 + n_z^2
	\end{equation}
	
	Usando \(\cos^2\alpha + \cos^2\beta + \cos^2\gamma = 1\):
	\begin{equation}
		\nu = \frac{c}{2a}\sqrt{n_x^2 + n_y^2 + n_z^2}
	\end{equation}
	
	Cada trío \((n_x,n_y,n_z)\) define un modo de oscilación único.
	

	Por tanto, el número de modos:
	\begin{equation}
		N(\nu)d\nu = \frac{1}{8}(4\pi r^2 dr) \times 2 = \frac{8\pi a^3}{c^3}\nu^2 d\nu
	\end{equation}
	
	Donde:
	\begin{itemize}
		\item \(r = \frac{2a}{c}\nu\): Radio de la esfera definida por $n_x^2 + n_y^2 + n_z^2$
		\item Factor \(\frac{1}{8}\): Solo primer octante (\(n_x,n_y,n_z > 0\))
		\item Factor 2: Dos polarizaciones posibles
	\end{itemize}
	
	\begin{equation}
		\rho_T(\nu)d\nu = \underbrace{\frac{8\pi\nu^2}{c^3}}_{\substack{\text{Densidad de} \\ \text{modos}}} \times \underbrace{kT}_{\substack{\text{Energía por} \\ \text{modo (equipartición)}}} d\nu
	\end{equation}
	
	\subsection{Deducción de la Fórmula de Planck usando el Ejemplo 1.4}
	
	\subsubsection*{Postulado de Energías Discretas}
	\textbf{Fuente}: Ecuación 1-26 del libro (p. 35).\\
	Planck propone que la energía de los osciladores en la cavidad solo puede tomar valores discretos:
	\begin{equation}
		\mathcal{E}_n = nh\nu \quad \text{con } n = 0,1,2,\ldots
	\end{equation}
	donde:
	\begin{itemize}
		\item \(h = 6.63 \times 10^{-34}\, \text{J$\cdot$s}\) (Constante de Planck)
		\item \(\nu\): Frecuencia de oscilación
	\end{itemize}
	
	\subsubsection*{Distribución de Boltzmann Modificada}
	\textbf{Fuente}: Ecuación 1-20 del libro (p. 32).\\
	La probabilidad de ocupar el estado \(n\) viene dada por:
	\begin{equation}
		P(n) = \frac{e^{-\mathcal{E}_n/kT}}{Z}
	\end{equation}
	donde:
	\begin{itemize}
		\item \(Z = \sum_{n=0}^\infty e^{-\mathcal{E}_n/kT}\) (Función de partición)
		\item \(k = 1.38 \times 10^{-23}\, \text{J/K}\) (Constante de Boltzmann)
		\item \(T\): Temperatura absoluta
	\end{itemize}
	
	\subsubsection*{Cálculo de la Función de Partición}
	Sustituyendo \(\mathcal{E}_n = nh\nu\):
	\begin{align}
		Z &= \sum_{n=0}^\infty e^{-nh\nu/kT} \nonumber \\
		&= 1 + e^{-h\nu/kT} + e^{-2h\nu/kT} + \cdots \nonumber \\
		&= \frac{1}{1 - e^{-h\nu/kT}} \quad \text{(Serie geométrica con razón } x = e^{-h\nu/kT} < 1)
	\end{align}
	
	\subsubsection*{Cálculo de la Energía Promedio}
	\textbf{Fuente}: Ecuación 1-21 del libro (p. 33).\\
	La energía promedio es:
	\begin{align}
		\langle \mathcal{E} \rangle &= \frac{1}{Z}\sum_{n=0}^\infty nh\nu e^{-nh\nu/kT} \nonumber \\
		&= h\nu \frac{\sum_{n=0}^\infty n e^{-nh\nu/kT}}{\sum_{n=0}^\infty e^{-nh\nu/kT}} 
	\end{align}
	
	\subsubsection*{Evaluación de las Series}
	Usando la identidad para series:
	\begin{equation}
		\sum_{n=0}^\infty n x^n = \frac{x}{(1 - x)^2} \quad \text{con } x = e^{-h\nu/kT}
	\end{equation}
	
	Sustituyendo en (5):
	\begin{align}
		\langle \mathcal{E} \rangle &= h\nu \frac{e^{-h\nu/kT}/(1 - e^{-h\nu/kT})^2}{1/(1 - e^{-h\nu/kT})} \nonumber \\
		&= h\nu \frac{e^{-h\nu/kT}}{1 - e^{-h\nu/kT}} \nonumber \\
		&= \frac{h\nu}{e^{h\nu/kT} - 1} \quad \text{(Ecuación clave de Planck)}
	\end{align}
	
	\subsubsection*{Densidad de Energía Espectral}
	\textbf{Fuente}: Combinación con densidad de modos de Rayleigh-Jeans (p. 30).\\
	La densidad de energía total se obtiene multiplicando por el número de modos:
	\begin{equation}
		\rho_T(\nu)d\nu = \underbrace{\frac{8\pi \nu^2}{c^3}}_{\substack{\text{Densidad de} \\ \text{modos de R-J}}} \times \underbrace{\frac{h\nu}{e^{h\nu/kT} - 1}}_{\substack{\text{Energía promedio} \\ \text{de Planck}} \times d\nu}
	\end{equation}
	
	\subsubsection*{Relación con Leyes Empíricas}
	\begin{itemize}
		\item \textbf{Ley de Wien (\(\nu \to \infty\))}: 
		\begin{equation}
			e^{h\nu/kT} \gg 1 \Rightarrow \rho_T(\nu) \approx \frac{8\pi h\nu^3}{c^3}e^{-h\nu/kT}
		\end{equation}
		
		\item \textbf{Ley de Stefan-Boltzmann}:
		\begin{align}
			R_T &= \int_0^\infty \frac{c}{4}\rho_T(\nu)d\nu \nonumber \\
			&= \frac{2\pi h}{c^2} \int_0^\infty \frac{\nu^3}{e^{h\nu/kT} - 1}d\nu \nonumber \\
			&= \sigma T^4 \quad \text{con } \sigma = \frac{2\pi^5 k^4}{15c^2 h^3}
		\end{align}
	\end{itemize}
	
	\subsubsection*{Tabla de Factos}
	\begin{tabular}{p{3cm}p{6cm}p{5cm}}
		\toprule
		\textbf{Paso} & \textbf{Justificación Matemática} & \textbf{Base Física} \\
		\midrule
		Cuantización & \(\mathcal{E}_n = nh\nu\) & Postulado de Planck (evitar catástrofe UV) \\
		Distribución & \(P(n) \propto e^{-\mathcal{E}_n/kT}\) & Estadística de Boltzmann \\
		Serie geométrica & \(\sum_{n=0}^\infty x^n = 1/(1-x)\) & \(x = e^{-h\nu/kT} < 1\) garantiza convergencia \\
		Energía promedio & \(\sum n x^n = x/(1-x)^2\) & Desarrollo matemático de series \\
		Límite Wien & \(e^{h\nu/kT} \gg 1\) & Comportamiento a altas frecuencias \\
		Integral Stefan & \(\int_0^\infty \frac{x^3}{e^x - 1}dx = \pi^4/15\) & Solución de integral estándar \\
		\bottomrule
	\end{tabular}
	
	\subsection*{Glosario de Términos}
	\begin{tabular}{p{2.5cm}p{4cm}p{3cm}}
		\toprule
		\textbf{Símbolo} & \textbf{Definición Conceptual} & \textbf{Unidades} \\
		\midrule
		\(\mathcal{E}_n\) & Energía del estado cuántico n & J \\
		\(Z\) & Función de partición (suma de estados) & Adimensional \\
		\(\langle \mathcal{E} \rangle\) & Energía promedio por modo & J \\
		\(\rho_T(\nu)\) & Densidad espectral de energía & J/m³Hz \\
		\(h\) & Cuanto elemental de acción & J$\cdot$s \\
		\(k\) & Constante de Boltzmann & J/K \\
		\(c\) & Velocidad de la luz & m/s \\
		\bottomrule
	\end{tabular}
	

	\subsection{Tabla Comparativa Detallada}
	\begin{table}[H]
		\centering
		\begin{tabular}{p{4cm}cc}
			\toprule
			\textbf{Aspecto} & \textbf{Rayleigh-Jeans} & \textbf{Planck} \\
			\midrule
			Tratamiento energético & Continuo & Cuantizado (\(\Delta E = h\nu\)) \\
			Distribución de energía & Equipartición (\(kT\)) & \(\frac{h\nu}{e^{h\nu/kT} - 1}\) \\
			Comportamiento a altas \(\nu\) & \(\rho_T \propto \nu^2\) (Catástrofe) & \(\rho_T \propto \nu^3 e^{-h\nu/kT}\) \\
			Relación con paredes & Nodos fijos (cond. frontera) & Mismo + cuantización energética \\
			Base matemática & Ondas estacionarias clásicas & Estadística cuántica de Bose \\
			\bottomrule
		\end{tabular}
	\end{table}
	
	\subsection{Resumen de Fórmulas y Glosario}
	\begin{table}[H]
		\centering
		\caption{Glosario de Variables y Fórmulas Clave}
		\begin{tabular}{llll}
			\toprule
			\textbf{Símbolo} & \textbf{Definición Conceptual} & \textbf{Unidades} & \textbf{Ecuación} \\
			\midrule
			\(\rho_T(\nu)\) & Energía almacenada por unidad de volumen y frecuencia & J/m³Hz & \(\frac{8\pi h\nu^3}{c^3}\frac{1}{e^{h\nu/kT} - 1}\) \\
			\(R_T\) & Potencia total emitida por unidad de área & W/m² & \(\sigma T^4\) \\
			\(\lambda_{\text{max}}\) & Longitud de onda de máxima emisión & m & \(\frac{hc}{4.965kT}\) \\
			\(\sigma\) & Constante de proporcionalidad en ley de Stefan & $W/m^2K^4$ & \(\frac{2\pi^5 k^4}{15c^2h^3}\) \\
			\(h\) & Cuanto elemental de acción & J$\cdot$s & \(6.63 \times 10^{-34}\) \\
			\(k\) & Constante de proporcionalidad energética & J/K & \(1.38 \times 10^{-23}\) \\
			\(c\) & Velocidad de propagación luminosa & m/s & \(3 \times 10^8\) \\
			\bottomrule
		\end{tabular}
	\end{table}
	
	\begin{itemize}
		\item \textbf{Ley de Stefan-Boltzmann}: 
		\begin{equation*}
			R_T = \sigma T^4\quad \left[\sigma = 5.67\times10^{-8}\,\text{W/m}^2\text{K}^4\right]
		\end{equation*}
		
		\item \textbf{Ley del desplazamiento de Wien}:
		\begin{equation*}
			\lambda_{\text{max}}T = 2.898\times10^{-3}\,\text{mK}
		\end{equation*}
	
		\item \textbf{Número de modos}
		\begin{equation}
			N(\nu)d\nu = \frac{8\pi a^3}{c^3}\nu^2 d\nu
		\end{equation}
		
		\item \textbf{Fórmula de Rayleigh-Jeans}:
		\begin{equation}
			\rho_T(\nu)d\nu = \frac{8\pi\nu^2}{c^3} kT d\nu \quad \left[\text{J/m}^3\text{Hz}\right]
		\end{equation}
		
		\item \textbf{Densidad de energía (Planck)}:
		\begin{equation*}
			\rho_T(\nu) = \frac{8\pi h\nu^3}{c^3}\frac{1}{e^{h\nu/kT} - 1}\quad \left[\text{J/m}^3\text{Hz}\right]
		\end{equation*}
	
		\item \textbf{Relación radiancia-densidad}:
		\begin{equation}
			R_T(\nu)d\nu = \frac{c}{4}\rho_T(\nu)d\nu
		\end{equation}
		
		\item \textbf{Constantes}:
		\begin{align*}
			h &= 6.63\times10^{-34}\,\text{J$\cdot$s} \quad \text{(Planck)}\\
			k &= 1.38\times10^{-23}\,\text{J/K} \quad \text{(Boltzmann)}\\
			c &= 3\times10^8\,\text{m/s} \quad \text{(Velocidad luz)}
		\end{align*}
	\end{itemize}
	
	
\end{document}