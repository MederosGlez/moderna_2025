\documentclass{article}
\usepackage[utf8]{inputenc}
\usepackage{amsmath}
\usepackage{enumitem}

\title{Preguntas y Respuestas sobre Cuerpo Negro y Temas Relacionados}
\author{}
\date{}

\begin{document}

\maketitle

\begin{enumerate}[leftmargin=*]

\item \textbf{¿Un cuerpo negro siempre se ve negro? Explique el término cuerpo negro.} \\
\textbf{Respuesta:} Un cuerpo negro es un objeto idealizado que absorbe toda la radiación electromagnética incidente, sin reflejar ni transmitir nada. A temperatura ambiente, un cuerpo negro sí se vería negro. Sin embargo, cuando se calienta, emite radiación térmica (radiación de cuerpo negro) cuyo color depende de su temperatura (e.g., rojo incandescente a altas temperaturas). Por lo tanto, no siempre se ve negro.

\item \textbf{En un fuego hecho con carbón, los huecos entre carbones se ven más brillantes. ¿La temperatura de dichos huecos es mayor que la de un carbón incandescente?} \\
\textbf{Respuesta:} No necesariamente. Los huecos actúan como cavidades de cuerpo negro, emitiendo radiación de manera más eficiente que la superficie rugosa del carbón. La mayor brillantez se debe a la emisividad cercana a 1 en las cavidades, no a una temperatura significativamente mayor.

\item \textbf{¿Es visible algún detalle del interior de una cavidad con paredes a temperatura constante?} \\
\textbf{Respuesta:} No. En equilibrio térmico, la radiación dentro de la cavidad es homogénea e isotrópica, por lo que no se distinguen detalles. Esto se conoce como radiación de cuerpo negro.

\item \textbf{¿Por qué no se usa \( R_T = \sigma T^4 \) para definir temperatura (e.g., 100°C)?} \\
\textbf{Respuesta:} La ley de Stefan-Boltzmann (\( R_T = \sigma T^4 \)) es válida solo para cuerpos negros ideales. En la práctica, los materiales tienen emisividades menores a 1, lo que introduce variaciones. No es una base universal para definir temperatura.

\item \textbf{¿Por qué un metal brilla a 1100 K, pero el cuarzo no?} \\
\textbf{Respuesta:} El cuarzo es transparente a la luz visible, por lo que no emite radiación térmica en ese rango. En cambio, el metal es opaco y emite radiación visible (color rojo) a 1100 K.

\item \textbf{Lista de funciones de distribución en ciencias sociales (variable discreta o continua):} \\
\textbf{Respuesta:}
\begin{itemize}
    \item \textbf{Normal (Gaussiana):} Continua (ej. ingresos en grandes poblaciones).
    \item \textbf{Binomial:} Discreta (ej. éxitos/fracasos).
    \item \textbf{Pareto:} Continua (distribución de ingresos altos).
    \item \textbf{Poisson:} Discreta (eventos raros en tiempo/espacio).
\end{itemize}

\item \textbf{Unidades de la constante entre radiancia espectral y densidad de energía:} \\
\textbf{Respuesta:} La relación es \( R_T(\nu) = \frac{c}{4} \rho_T(\nu) \). La constante \( \frac{c}{4} \) tiene unidades de velocidad (\( \text{m/s} \)).

\item \textbf{Origen de la catástrofe ultravioleta:} \\
\textbf{Respuesta:} Surge de la ley de Rayleigh-Jeans, que predice una densidad de energía infinita a longitudes de onda cortas (\(\lambda \to 0\)). Esto se resolvió con la teoría cuántica de Planck, que introduce la cuantización de la energía.

\item \textbf{Relación entre fallas de la equipartición en calores específicos y radiación:} \\
\textbf{Respuesta:} Ambas fallas surgen al asumir energía continua. La equipartición no explica calores específicos de sólidos a bajas temperaturas (Einstein/Debye) ni la radiación de cuerpo negro (requiere cuantización).

\item \textbf{Comparación de radiancia espectral (\( R_T(\nu) \)), radiancia (\( R_T \)) y densidad de energía (\( \rho_T(\nu) \)):} \\
\textbf{Respuesta:}
\begin{itemize}
    \item \textbf{Radiancia espectral (\( R_T(\nu) \)):} Potencia por unidad de área, frecuencia y ángulo sólido. Unidades: \( \text{W}/(\text{m}^2 \cdot \text{Hz} \cdot \text{sr}) \).
    \item \textbf{Radiancia (\( R_T \)):} Integral de \( R_T(\nu) \) en todas las frecuencias y ángulos. Unidades: \( \text{W}/\text{m}^2 \).
    \item \textbf{Densidad de energía (\( \rho_T(\nu) \)):} Energía por unidad de volumen y frecuencia. Unidades: \( \text{J}/(\text{m}^3 \cdot \text{Hz}) \).
\end{itemize}


\item \textbf{¿Por qué se usa pirometría óptica por encima del punto de fusión del oro? ¿En qué objetos se aplica?} \\
\textbf{Respuesta:} La pirometría óptica mide la radiación térmica en el espectro visible. Por encima de 1064°C (punto de fusión del oro), los materiales emiten suficiente luz visible para mediciones precisas. Se usa en metalurgia, hornos industriales y procesos de alta temperatura.

\item \textbf{¿Existen cantidades cuantizadas en física clásica? ¿Está cuantizada la energía clásicamente?} \\
\textbf{Respuesta:} Sí, en sistemas como ondas estacionarias (frecuencias discretas). Sin embargo, la energía en física clásica es continua, no cuantizada.

\item \textbf{¿Tiene sentido hablar de cuantización de carga? ¿En qué difiere de la energía?} \\
\textbf{Respuesta:} Sí, la carga está cuantizada (e.g., \( \pm e \)). La diferencia es que la carga tiene un valor fundamental (\( e \)), mientras la energía tiene múltiples niveles discretos.

\item \textbf{¿La masa en reposo de partículas elementales es cuantizada?} \\
\textbf{Respuesta:} Las masas son discretas, pero no se considera "cuantización" como en carga o energía, pues no hay una teoría que lo establezca.

\item \textbf{Sistemas clásicos con frecuencias cuantizadas. ¿Energía cuantizada?} \\
\textbf{Respuesta:} Ejemplos: cuerdas vibrantes, cavidades resonantes. Las frecuencias son discretas, pero la energía clásica asociada es continua.

\item \textbf{Demuestre que \( h \) tiene dimensiones de momento angular. ¿Implica cuantización?} \\
\textbf{Respuesta:} \( [h] = \text{J} \cdot \text{s} = \text{kg} \cdot \text{m}^2/\text{s} \), igual que momento angular. No implica cuantización universal, aunque en QM sí se cuantiza.

\item \textbf{Orden de magnitud mínimo de \( h \) para efectos cuánticos cotidianos.} \\
\textbf{Respuesta:} Si \( h \sim 1 \, \text{J} \cdot \text{s} \), efectos cuánticos serían macroscópicos (e.g., objetos visibles con comportamiento ondulatorio).

\item \textbf{Radiación de cuerpo negro a 3 K y temperatura del espacio.} \\
\textbf{Respuesta:} Corresponde a la radiación cósmica de fondo (CMB) a ~2.7 K, indicando que el espacio tiene una temperatura cercana al cero absoluto.

\item \textbf{¿La teoría de Planck sugiere energía atómica cuantizada?} \\
\textbf{Respuesta:} Sí, Planck propuso cuantización de osciladores en cuerpos negros, sentando bases para la cuantización atómica (Bohr).

\item \textbf{¿Por qué se descubrió la cuantización en espectros continuos de sólidos?} \\
\textbf{Respuesta:} En sólidos, las interacciones entre átomos generan un espectro casi continuo, pero las desviaciones (ley de radiación de Planck) revelaron la necesidad de cuantización. En gases, las líneas discretas ya eran conocidas, pero no exigían una teoría cuántica.
\end{enumerate}

\end{document}